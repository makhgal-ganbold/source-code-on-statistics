\documentclass[11pt]{letter}
\usepackage[utf8x]{inputenc}
\usepackage[mongolian]{babel}
\usepackage[a4paper,margin=2cm,bottom=2.5cm]{geometry}
\usepackage[colorlinks=true,linkcolor=blue,urlcolor=blue,citecolor=blue,anchorcolor=blue]{hyperref}
\usepackage{listings,xcolor,parskip}

\lstset{
  language=R,
  basicstyle={\ttfamily},
  %backgroundcolor=\color{black!15},
  keywordstyle=\color{blue!85},
  commentstyle=\color{red!85},
  stringstyle=\color{black!75},
  breakatwhitespace=true,
  breaklines=true,
  frame=l,
  framesep=5pt,framerule=1pt,
  xleftmargin=7pt,xrightmargin=0pt,
  rulecolor=\color{blue!65},
  keepspaces=true,
  showstringspaces=false,
  columns=flexible,
  numbers=none,
  tabsize=3,
  morekeywords={in},
  otherkeywords={},
  deletendkeywords={},
  deletekeywords={R}
}

\begin{document}
\begin{center}
\bfseries
\Large Code Battle against Students in R \\
\large Battle \#1 Fall Semester 2019
\end{center}
\begin{enumerate}
\item Өгсөн тоо төгс тоо эсэхийг шалгана уу. Тоо өөрөөсөө бусад хуваагчдынхаа нийлбэртэй тэнцүү бол түүнийг төгс тоо гэнэ.
\par
\lstinline[otherkeywords={is.perfect}]|is.perfect(x)| функц зарлана. \lstinline|x| төгс тоо бол \lstinline|TRUE|, эсрэг тохиолдолд \lstinline|FALSE| утга буцаана. Функцийг дараах тестээр шалгана.
\begin{lstlisting}[otherkeywords={is.perfect,expect_true,expect_false}]
testthat::expect_true(is.perfect(496))
testthat::expect_false(is.perfect(25))
\end{lstlisting}
Эх сурвалж: \href{https://www.britannica.com/science/perfect-number}{www.britannica.com/science/perfect-number}
\item $e$ тоо олох програм зохионо уу. $$e=1+\frac{1}{1!}+\frac{1}{2!}+\frac{1}{3!}+\frac{1}{4!}+\ldots+\frac{1}{n!}$$
\par
\lstinline[otherkeywords={number.e}]|number.e(n)| функц зарлана. Тооцоонд дээрх томъёог ашиглана. Функцийг дараах тестээр шалгана.
\begin{lstlisting}[otherkeywords={number.e,expect_equivalent}]
testthat::expect_equivalent(object = number.e(10), expected = 2.7182818284590452353602874713527)
\end{lstlisting}
Эх сурвалж: \href{https://www.mathsisfun.com/numbers/e-eulers-number.html}{www.mathsisfun.com/numbers/e-eulers-number.html}
\item $1,2,3,\ldots,n$ тоонуудын цифрүүдийн нийлбэрийг олно уу.
\par
\lstinline[otherkeywords={digit.sum}]|digit.sum(n)| функц зарлана. Функцийг дараах тестээр шалгана.
\begin{lstlisting}[otherkeywords={digit.sum,expect_equivalent}]
testthat::expect_equivalent(object = digit.sum(13), expected = 55)
testthat::expect_equivalent(object = digit.sum(20), expected = 102)
\end{lstlisting}
Эх сурвалж: \href{https://rechneronline.de/digit-sum/}{rechneronline.de/digit-sum/}
\item $A(n,m)$ хэмжээст хүснэгт өгчээ. Сондгой дугаартай мөрүүдийн арифметик дунджийг олно уу.
\par
\lstinline[otherkeywords={odd.rowMeans}]|odd.rowMeans(x)| функц зарлана. \lstinline|x| аргумент матриц хэлбэртэй байна. Харин функцийн буцаах утга вектор хэлбэртэй байна. Функцийг дараах тестээр шалгана.
\begin{lstlisting}[keywords={c},otherkeywords={odd.rowMeans,read.table,expect_silent,expect_equivalent}]
testthat::expect_silent({
  X <- read.table(text = "
    9 5 6 5
    8 6 1 9
    8 1 7 3
    7 8 4 9
    5 2 9 7")
  Y <- read.table(text = "
    9
    8
    8
    7
    5")
})
testthat::expect_equivalent(object = odd.rowMeans(X), expected = c(6.25, 4.75, 5.75))
testthat::expect_equivalent(object = odd.rowMeans(Y), expected = c(9, 8, 5))
\end{lstlisting}
\item Өгсөн тэмдэгт мөр палиндром эсэхийг шалгана уу.
\par
\lstinline[otherkeywords={is.palindrome}]|is.palindrome(x)| функц зарлана. \lstinline|x| тэмдэгт мөр палиндром бол \lstinline|TRUE|, эсрэг тохиолдолд \lstinline|FALSE| утга буцаана. Функцийг дараах тестээр шалгана.
\begin{lstlisting}[otherkeywords={is.palindrome,expect_true,expect_false}]
testthat::expect_true(is.palindrome("civic"))
testthat::expect_true(is.palindrome("madam"))
testthat::expect_true(is.palindrome("refer"))
testthat::expect_false(is.palindrome("abc"))
testthat::expect_false(is.palindrome("baba"))
testthat::expect_false(is.palindrome("Madam"))
\end{lstlisting}
Эх сурвалж: \href{https://en.wikipedia.org/wiki/Palindrome}{en.wikipedia.org/wiki/Palindrome}
\end{enumerate}
\end{document}
